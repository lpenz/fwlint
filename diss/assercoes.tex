


\section{Introduction}

Up to this point, PFC is able to detect contradictions and other
problems in an algebraic way, considering the set of rules implemented and the
current network topology.

It is interesting to provide a mechanism to perform a
functional verification of the filter, that is, to test if the filter in fact
conforms to its specification.

This is done with \emph{assertions}.



\section{Requirements}

The set of assertions will also be provided as input to PFC, in addition to the
set of rules and the network topology.

Assertions will be simple verifications of the overall filter behaviour. They
are checked after the algebraic verifications, and their violation is reported with
the anomalies.



\section{Model}

Each assertion has the same format of a rule, being composed of a match and a
target.
\begin{equation*}
	assertion =
	\begin{cases}
		match & \in Regionsets \\
		action & \in \{ \mbox{accept}, \mbox{deny} \}
	\end{cases}
\end{equation*}

Unlike rules, though, assertions must be globally true, all of them,
simultaneously. Assertions must hold in the global accessibility profile,
as that guarantees that assertions hold globally when there are no anomalies.



\section{Errors}

As assertions are not part of the implemented topology, they do not generate
anomalies, but \defpar{assertions errors}.

Assertions are first checked among themselves. As all assertions must hold
globally, the error of
\defpar[assertion!]{contradiction} is reported for assertions that contradict one
another.

If an assertion does not hold, an assertion \defpar[assertion!]{violation} error is
reported.



\section{Algorithm}

Assertion verification is performed the same way rule disagreement is checked.
See algorithm \ref{alg:checkonepath} for the high-level reference.



\section{Conclusion}

Assertions are a simple mechanism for filter conformance checking.

They can also help evaluate the impact of a specification change, help
with its implementation and help in regression testing.

Assertions can be compared to unit testing for packet filter configuration.



% vim: ft=tex spelllang=en

