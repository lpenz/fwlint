
\newcommand{\myrule}[1]{{\tt{#1}}}
\newcommand{\myfilter}[1]{{\tt{#1}}}
\newcommand{\myassertion}[1]{{\tt{#1}}}
\newcommand{\mynet}[1]{{\tt{#1}}}

{\color{white} \cite{knuth1}}

\section{Introduction}

A case study is presented in this chapter.

The case study is an artificial example. As the most interesting and non-obvious
problems of a network arise in the presence of changes, the case study is
divided in iterations. Each iteration changes the network and the requirements
before implementing a na�ve solution that will be checked with PFC and then fixed.

Assertions are used along a security policy to verify the implementation's
conformance.



\section{Case 1: single filter}

Misc. Inc. started out as a small company. At the beginning, we had only a small
handful of computers connected to a switch. As the company grew, security became
an issue. After a virus wiped out our customer's information, the company decided to
specialize roles, split the network and develop a security policy.

The network got divided by functionality and access requirements. Servers were put in
two networks: one for the IT servers (mail, file, proxy, web server) and one for the
main servers. Users were put in a third network.
The topology can be seen in figure \ref{fig:case1naive:topo}.

The security policy stated that:
\begin{itemize}
	\item every network has access to the mail server (IT);
	\item internet web access is filtered by a proxy (IT) that can access the
		internet;
	\item the web server (IT) can be accessed from the internet;
	\item the other servers should not be able to initiate connections;
	\item users have complete access to the main servers and the file server;
	\item users can only access the internet through the proxy.
\end{itemize}

\begin{figure}
	\imagenorm{case1naive.pdf}
	\caption{\label{fig:case1naive:topo}Topology of case 1}
\end{figure}



\subsection{Na�ve solution}

The IP address allocation follows. Networks have their first letter capitalized;
server networks and hosts end in an {\tt{S}}; filters end in {\tt{F}} and routers
in {\tt{R}}.

\input{case1naive_aliases.tex}

The direct translation of the security policy results in the following
description:

\input{case1naive.tex}

Figure \ref{fig:case1naive:profile} shows the profile of the filter, figure
\ref{fig:case1naive:tree} shows the tree of rules.

\begin{figure}
	\imagefit{case1naive_profile.pdf}
	\caption{\label{fig:case1naive:profile}Profile of the na�ve solution of case 1}
\end{figure}

\begin{figure}
	\imagefit{case1naive_ruletree.pdf}
	\caption{\label{fig:case1naive:tree}Tree of rules of the na�ve solution of case 1}
\end{figure}


PFC found the following anomalies:


\verbatiminput{case1naive.res}


It is easy to see that the invisibility of rule \myrule{Users2Proxy} and its
consequent irrelevancy is due to the fact that the file server and proxy server
are the same physical machine, and the more general rule (\myrule{Users2File})
comes first in the filter.

On the other hand, the conflicts with the \myrule{NoServer} rule happen because
the rules \myrule{Mail} and \myrule{Web} allow the main servers to connect to
the mail and web servers, and that is against the security policy.

At last, the redundancy of rule \myrule{NoServer} is pointed out because its
effect is already provided by the \myrule{Default} rule.

In the next section a proper solution is developed.



\subsection{Proper solution}

A simple fix for the problems found in the na�ve solution is to remove the rules
\myrule{Users2Proxy} and \myrule{NoServer}. Even though the first rule may be
safely removed, the second rule points to a real conflict between what was
implemented and the security policy.

A better approach is to transform both rules into \emph{assertions}, as they are
both correct and must be globally valid: the first rule is invisible only because the
file server and the proxy server are accessed through the same IP address; the
second rule is really redundant, but verifies the filter implementation.

After converting both rules to assertions, the assertion \myassertion{NoServer}
fails. To fix that, three rules are implemented: \myrule{NoServer2Mail} and
\myrule{NoServer2Web}, to prevent all servers from accessing the mail and web
server and to prevent the proxy server from accessing the main server. The
output of PFC is then:

\verbatiminput{case1proper.res}

It's interesting to note that the isolated analysis of this filter would point
out that the rule \myrule{NoProxy2Server} conflicts with the rules \myrule{Mail} and
\myrule{Web}. However, these conflicts are filtered out as the conflicting
datagrams are all within a single interface and would not go through the filter.

The final rule set is:

\input{case1proper.tex}

Figure \ref{fig:case1proper:profile} show the profile of this solution. The
tree of rules of the proper solution ca be seen in figure
\ref{fig:case1proper:tree}.

\begin{figure}
	\imagefit{case1proper_profile.pdf}
	\caption{\label{fig:case1proper:profile}Profile of the proper solution of case 1}
\end{figure}

\begin{figure}
	\imagefit{case1proper_ruletree.pdf}
	\caption{\label{fig:case1proper:tree}Tree of rules of the proper solution of case 1}
\end{figure}




\section{Case 2: Multiple filters}

That solution has worked very well until we had a critical failure on
a component that had to be replaced. As we were in a position where
network downtime was very expensive, we decided that the best course of action
was to increase redundancy and to buy 3 filters. Those would be placed in a
security-minded arrangement, with a replacement priority; that is, if the
internal filter failed, it would be replaced with the internet filter while the
faulty filter was fixed. The new topology can be seen in figure
\ref{fig:case2naive:topo}.

\begin{figure}
	\imagefit{case2naive_dot.pdf}
	\caption{\label{fig:case2naive:topo}Topology of case 2}
\end{figure}

\subsection{Na�ve solution}

To make a safe transition, the IT department decided to replicate the old
filter's rules in all filters and remove the rules that were related to networks
not directly connected. They also noticed that assertions were a good thing and
coded some more. The initial setup can be seen below:

\input{case2naive.tex}

\begin{figure}
	\imagefit{case2naive_profile.pdf}
	\caption{\label{fig:case2naive:profile}Profile of the na�ve solution of case 2}
\end{figure}

PFC found the following problems:

\verbatiminput{case2naive.res}

Although the solution implements the desired behaviour, it is nowhere
near minimal, and has some other issues.

The first problems are the disagreements. They are reported because even though
\myfilter{ITUsersF} is not connected to \mynet{MainS}, it must agree with
\myfilter{InternalF} on whether the mail and web servers can connect to the main
servers. To fix it, we add \myrule{NoServer2Mail} and \myrule{NoServer2Web} to
\myfilter{ITUsersF}.

Following the disagreements, some irrelevant rules are pointed out. It is safe
to remove the corresponding rules.

The last problem is the assertion failure. It happens because, as in the
single filter case, the proxy and the file server are the same PC. This time it
is better to split them for the sake of security than remove the assertion.



\subsection{Proper solution}

After removing the irrelevant rules, the following result is generated:

\verbatiminput{case2proper0.res}

Besides the redundant rules, the failed assertions shows that now that the file
server and proxy are not on the same machine, an explicit rule for the proxy is
needed. We remove the redundant rules, and add the proxy rule to get:

\verbatiminput{case2proper1.res}

The profile is displayed in figure \ref{fig:case2proper1:profile}. It is easy to
see that the profile is equivalent to the final profile of the single filter
case (figure \ref{fig:case1proper:profile}), except for the proxy/file
server split.

The final setup is below:

\begin{figure}
	\imagefit{case2proper1_profile.pdf}
	\caption{\label{fig:case2proper1:profile}Profile of the proper solution of case 2}
\end{figure}

\input{case2proper1.tex}

By increasing the number of filters with a carefully thought topology, we have
not only increased security but also decreased the complexity of each filter,
making maintenance easier.



\section{Case 3: Even more redundancy}

Even though the current setup allows a filter to fail without compromising the
whole network, it was decided that even more redundancy was desired. The
extreme filters \myfilter{InternetF} and \myfilter{InternalF} were then
connected, and a router between \mynet{Users} and \mynet{MainS} was deployed.
The resulting topology can be seen in figure \ref{fig:case3naive:topo}.

\begin{figure}
	\imagefit{case3naive.pdf}
	\caption{\label{fig:case3naive:topo}Topology of case 3}
\end{figure}

This setup prevents the loss of connectivity between \mynet{Users} and the
internet when the filter \myfilter{ITUsersF} fails and between \mynet{Users} and
\mynet{MainS} when the filter \myfilter{InternalF} fails.



\subsection{Na�ve solution}

Keeping the current rule set generates the profile in figure
\ref{fig:case3naive:profile}. PFC reports:

\verbatiminput{case3naive.res}

Now that \myfilter{InternalF} is connected to \myfilter{InternetF}, they must
agree on the profile of the IT servers network. The last 7 disagreements are
reported because of this. The other disagreement is reported because
\myfilter{ITUsersF} does not allow the proxy to access the internet. With the new
connection, that is a possible path, and will be used if the port of
\myfilter{InternetF} that connects it to \mynet{ITS} fails.

Moving on to the blocks, it is easy to see that they are reported because the
filters do now allow the traffic that previously would not go through them:
\myfilter{InternalF} must support IT and internet traffic;
\myfilter{ITUsersF} and \myfilter{InternetF} must support the traffic to
\mynet{MainS}.

On the other hand, the leak reported was caused by the presence of a router in
parallel with \myfilter{InternalF}. That filter did not allow traffic from the
servers into the user network, but the router does. The only solution in this
case is to replace the router with a filter or eliminate the router completely.
We will do the former.

At last, the assertion problem was caused by the disagreements, that prevent
proper assertion evaluation.

\begin{figure}
	\imagefit{case3naive_profile.pdf}
	\caption{\label{fig:case3naive:profile}Profile of the na�ve solution of case 3}
\end{figure}



\subsection{Proper solution}

Now, there is so much redundancy, that almost every path is a possible one. The
best course of action is to replicate the rules of the single filter case in
every filter, including the new \myfilter{InternalF2} which replaced
\myfilter{InternalR}. That makes PFC give us:

\verbatiminput{case3proper0.res}

That, of course, happened because the IP address of the proxy was changed in
case 2. Adding the rule \myrule{Users2Proxy} to every filter, we get from PFC:

\verbatiminput{case3proper1.res}

The profile can be seen in figure \ref{fig:case3proper1:profile}, the topology is
in figure \ref{fig:case3proper1:topo}, while the complete setup is below.

\input{case3proper1.tex}

\begin{figure}
	\imagefit{case3proper1_profile.pdf}
	\caption{\label{fig:case3proper1:profile}Profile of the proper solution of case 3}
\end{figure}

\begin{figure}
	\imagefit{case3proper1.pdf}
	\caption{\label{fig:case3proper1:topo}Topology of the proper solution of case 3}
\end{figure}


% vim: ft=tex spelllang=en

